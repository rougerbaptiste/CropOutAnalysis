\documentclass[10pt,a4paper]{article}
\usepackage[utf8]{inputenc}
\usepackage[french]{babel}
%\usepackage{amsmath}
%\usepackage{amsfonts}
%\usepackage{amssymb}
%\usepackage{graphicx}
%\usepackage{subcaption}
\usepackage[left=2.5cm, right=2.5cm, top=2.5cm, bottom=2.5cm]{geometry}

\title{Manual for Analysis scripts of \textsc{CropMetaPop} Sensibility Analysis:\\ \textbf{newAnalysis.py}}
\author{Baptiste \textsc{Rouger}}
\begin{document}
\maketitle
\section{General purpose}
This script is made to analyse the GenotypeMono.csv file produced by \textsc{CropMetaPop} and return several genetic indices: $\overline{H}s$, $\overline{H}obs$, $\overline{F}is$, $Ht$ and Gst.\\
In a close future, it will put all these informations in an output file on which would be performed the sensibility analysis.\\

It takes a few command-line parameters in input:
\begin{enumerate}
    \item the path to the folder containing the GenotypeMono.csv file (e.g. /home/\$User/Folder/)
    \item the number of replicates of the simulation
    \item the number of populations in the simulation
    \item the number of markers in the populations
    \item the number of alleles of the markers

\end{enumerate}
\section{Protocol}
\begin{enumerate}
        \item Get all the values for the parameters of the script (this step will be done automatically for the sensibility analysis by the \texttt{setGenExpPlanX.py} script that will create corresponding launcher file for the analysis)
        \item Run the script using Python 3 (\texttt{python3 newAnalysis Path/to/Folder Replicates Populations Markers Alleles}) by replacing the parameters by their respective values
        \item TO DO: The result file is created in the folder specified by the user
\end{enumerate}

\end{document}
