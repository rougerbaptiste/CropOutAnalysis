\documentclass[10pt,a4paper]{article}
\usepackage[utf8]{inputenc}
%\usepackage[french]{babel}
%\usepackage{amsmath}
%\usepackage{amsfonts}
%\usepackage{amssymb}
%\usepackage{graphicx}
%\usepackage{subcaption}
%\usepackage[left=2.00cm, right=2.00cm, top=2.00cm, bottom=2.00cm]{geometry}

\title{Manual for Analysis scripts of \textsc{CropMetaPop} Sensibility Analysis}
\author{Baptiste \textsc{Rouger}}
\begin{document}
\maketitle
\section{General purpose}
\section{List of the files}
\begin{itemize}
        \item \textbf{expPlan.R} Script that generates the experiment plan
        \item \textbf{setGenExpPlanA.py} Script that uses the experiment plan (5 parameters) for experiment A (drift + selection)
        \item \textbf{setGenExpPlanB.py} Script that uses the experiment plan (9 parameters) for experiment B (drift + selection + colonisation)
        \item \textbf{setGenExpPlanC.py} Script that uses the experiment plan (9 parameters) for experiment C (drift + selection + migration)
\end{itemize}

\section{Protocol}
\begin{enumerate}
        \item Modify the file \textbf{expPlan.R} to the desired number of parameters and levels of parameters depending on the experiment. You can also modify the name of the output file (MyData.csv by default).
    \item Launch \textbf{expPlan.R} using \texttt{Rscript expPlan.R}. It creates the file containing the fractional experiment plan.
    \item Launch the python script corresponding to the experiment after modifying the name of the input file containing the experiment plan (MyData.csv) using \texttt{python setGenExpPlan}X\texttt{.py} with X the experiment to create.
    \item This creates a set of settings files for the \textsc{CropMetaPop} model (ending with \textsc{.set}), along with a file called \textbf{launcher} that contains the commands to launch the experiments using \texttt{condor\_submit}, and a \textbf{launcherR} file that contains the corresponding command to launch the analysis using \texttt{condor\_submit}.
\end{enumerate}

\end{document}
