\documentclass[10pt,a4paper]{article}
\usepackage[utf8]{inputenc}
\usepackage[french]{babel}
%\usepackage{amsmath}
%\usepackage{amsfonts}
%\usepackage{amssymb}
%\usepackage{graphicx}
%\usepackage{subcaption}
\usepackage[left=2.5cm, right=2.5cm, top=2.5cm, bottom=2.5cm]{geometry}

\title{Manual for Analysis scripts of \textsc{CropMetaPop} Sensibility Analysis:\\ \textbf{newAnalysis.py}}
\author{Baptiste \textsc{Rouger}}
\begin{document}
\maketitle
\section{General purpose}
These scripts are made to analyse the GenotypeMono.csv file produced by \textsc{CropMetaPop} and return several genetic indices: $\overline{H}s$, $\overline{H}obs$, $\overline{F}is$, $Ht$ and Gst.\\
It outputs each of these indices in .res files.\\

The script called \texttt{analysisNSel.py} analyses the neutral markers, while the script called \texttt{analysisSel.py} analyses the selected markers.\\

It takes a few command-line parameters in input:
\begin{enumerate}
    \item the path to the folder containing the GenotypeMono.csv file (e.g. /home/\$User/Folder/)
    \item the number of replicates of the simulation
    \item the number of populations in the simulation
    \item the number of markers in the populations
    \item the number of alleles of the markers

\end{enumerate}

Be careful, you cannot give to \texttt{analysisSel.py} a number of markers lower than 11!

\section{Protocol}
\begin{enumerate}
        \item Get all the values for the parameters of the script (this step will be done automatically for the sensibility analysis by the \texttt{setGenExpPlanX*Sel.py} script that will create corresponding launcher file for the analysis)
        \item Run the script using Python 3 (\texttt{python3 analysis*Sel.py Path/to/Folder Replicates Populations Markers Alleles}) by replacing the parameters by their respective values
        \item The result files are created in the folder of the simulation
\end{enumerate}

\end{document}
